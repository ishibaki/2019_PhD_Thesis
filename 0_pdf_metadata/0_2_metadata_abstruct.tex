---
abstract: |
    How left-right (LR) asymmetry forms in the animal body is a fundamental problem in Developmental Biology.
    While the mechanisms for LR asymmetry are well studied in some species, they are still poorly understood in invertebrates.
    We previously showed that the intrinsic LR asymmetry of cells (designated as cell chirality) drives LR asymmetric development in the *Drosophila* embryonic hindgut, although the machinery of the cell chirality formation remains elusive.
    Here, we found that the *Drosophila* homolog of the *Id* gene, *extra macrochaetae* (*emc*), is required for the normal LR asymmetric morphogenesis of this organ.
    Id proteins, including Emc, are known to interact with and inhibit E-box-binding proteins (E proteins), such as *Drosophila* Daughterless (Da).
    We found that the suppression of *da* by wild-type *emc* was essential for cell chirality formation and for normal LR asymmetric development of the embryonic hindgut.
    *MyosinID* (*MyoID*), which encodes the *Drosophila* Myosin ID protein, is known to regulate cell chirality.
    We further showed that Emc-Da regulates cell chirality formation, in which Emc functions upstream of or parallel to MyoID.
    Abnormal Id-E protein regulation is involved in various human diseases.
    Our results suggest that defects in cell shape may contribute to the pathogenesis of such diseases.
...

